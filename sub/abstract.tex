\documentclass[../main]{subfiles}
\begin{document}

\begin{abstract}
  上市公司是否实施高转股对中小投资者利益最大化有重要意义。本文着眼于此问题,
  通过经济学和统计学方法进行分析,得出了可以根据历史数据在一定程度上预测未来
  上市公司是否实施高转股的结论,从而验证了股票股利的交易区间理论。

  对采集到的数据进行整合后,先经插值处理,解决了缺失值问题后,通过主成分分析
  的方法,在保证主成分贡献率大于85 %的情况下,选择了对上市公司实施高送转有较
  大影响的因子。

  通过对比 Logistic 回归模型和支持向量机 SVM 模型以及同时使用这2种模型、 0--1
  标准化和均值方差标准化和不进行标准化 3 种情况对高送转股票进行预测得
  到的结果,分析了最优的预测方法。

  \begin{keyword}
    主成分分析,Logistic回归,支持向量机
  \end{keyword}
\end{abstract}

\begin{abstract*}
  It is of great significance to maximize the interests of small and
  medium-sized investors whether the listed companies implement the high
  conversion. This paper focuses on this problem, through the analysis of
  economic and statistical methods, draws a conclusion that we can predict
  whether the listed companies will implement the high conversion of shares
  to some extent according to the historical data, thus verifying the trading
  interval theory of stock dividends.

  After the integration of the collected data, first through interpolation
  processing, the problem of missing value is solved, and then through the
  method of principal component analysis, under the condition that the
  contribution rate of principal component is greater than 85\%, the factors
  that have a greater impact on the implementation of high transmission of
  listed companies are selected.

  Through the comparison of logistic regression model and SVM model, as well
  as the use of these two models, 0--1 standardization, mean variance
  standardization and no standardization, the paper analyzes the optimal
  prediction method.

  \begin{keyword*}
    Principal component analysis, Logistic regression, Support vector machine
  \end{keyword*}
\end{abstract*}

\end{document}

