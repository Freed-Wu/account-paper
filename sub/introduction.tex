\documentclass[../main]{subfiles}
\begin{document}

\chapter{绪论}%
\label{cha:introduction}

\section{问题背景}%
\label{sec:backend}

主板以及中小企业板的高送转股票在近几年日益盛行,导致了创业板公司更加注重高送
转。但信息不对称现象使得高送转股票容易涉及内幕交易。包括:

\begin{itemize}
  \item 拉抬上市公司股价,为上市公司的增发做铺垫;
  \item 帮助套牢机构实施突围;
  \item 掩护解禁股套现。
\end{itemize}

经证监会披露,高送转占涉及内幕交易的案件的比例高达3成以上。即便在证监会对涉嫌
市场操纵、内幕交易等违法行为的高送转公司加大了立案调查的力度的严厉打击下,凭
借自身掌握或获取的信息优势谋取非法利益, 违背了诚信履行受托责任,义务,严重破
坏资本市场信息公平原则和影响证券市场价格发现和资源配置功能的内幕交易人员仍旧
层出不穷。这是证监会加强市场监管工作的一个重大考验——证监会能否继续维持对内幕
交易等违法违规行为的零容忍。

在这种背景下,分析公司分配高送转股票股利的动机,研究市场和投资者对于更多的分
配股利的公司、更高比例的送转股水平的态度,有利于监管部门合力预期和把握后续对
市场新政的制定和出台有着至关重要的意义
。\cite{卢玉梅2014我国上市公司“高送转”行为动机实证研究}

\section{相关定义}%
\label{sec:definition}

\subsection{股利}%
\label{sec:dividend}

公司向股东分配的公司盈余。在全球各个国家的相关法律规定几乎相同
。\cite{孔小文2003上市公司股利政策信号传递效应的实证分析}

\begin{itemize}
  \item 在欧美国家,股利只能从公司历年盈余中支付。公司只有在具备足够的累积盈
    余时才能支付股利;
  \item 在我国,法律允许将超额面值缴入资本,也即资本公积(Additional Paid-in
    Capital)列入可供股东分配的范围。
\end{itemize}

公司股东大会或董事会制定对一切与股利有关的事项,即股利政策,包括:

\begin{itemize}
  \item 公司是否发放股利,即对其收益进行分配还是留存以用于再投资的策略问题;
  \item 股利的支付方式;
  \item 发放多少股利;
  \item 何时发放股利。
\end{itemize}

在我国,上市公司股利支付方式有多种:

\subsection{现金股利}%
\label{sub:cash_dividend}

以现金方式向股东发放的红利,最常见。必须具备条件:

\begin{itemize}
  \item 有足够的留存收益,以保证再投资资金的需要;
  \item 有足够的现金,以保证生产经营和支付股利;
  \item 有利于改善公司的财务状况
  \item 有董事会的决定。
\end{itemize}

现金红利的支付会直接影响股票的市场价格,公司必须依据实际情况权衡利弊,制
定合理的现金红利政策。

\subsection{股票股利}%
\label{sub:stock_dividend}

以增发股票作为股利。按原有普通股东的持股比例分配新股,其本质是等比例增加原有
股东持有流通股的数量,稀释每股普通股的权益,而不会影响公司的资产负债,也不会
增加股东权益的总额。股票股利实质是股东权益的内部结构调整,对净资产收益率没有
影响,对公司的盈利能力也并没有任何实质性影响。根据支付股票股利的资产来源,股
票股利又分为:

\begin{description}
  \item[送红股]将净利润以发放股票的形式分配给股东,结果是利润转为股本。送红股
    后,公司的资产、负债、股东权益的总额及结构并没有发生改变,但总股本增大了
    ,同时每股净资产降低了;
  \item[转增股票]将盈余公积金和资本公积金转换为股本,并以发放股票的形式分配给
    所有股东。盈余公积金是指企业按照规定从税后利润中提取的积累资金,主要用来
    弥补企业以往年度的亏损和转增资本。资本公积金是在公司的生产经营之外,由资
    本、资产本身及其他原因形成的股东权益收入。公司的资本公积金主要来源于的股
    票发行的溢价收入、接受的赠与、资产增值、因合并而接受其他公司资产净额等。
\end{description}

高送转指上市公司大比例送红股或大比例以资本公积金转增股票,一般每10股送红股和
转增股票大于等于5 。

\section{文献综述}%
\label{sec:document}

投资者是资本市场重要的参与者,保护投资者利益、积极回报投资者是资本市场的立身
之本。在成熟的资本市场中,完善的投资者法律保护机制、有效的市场约束机制以及较
为健全的公司治理机制,为保护股东利益提供了强有力的支撑。这些经济体中的上市公
司大多具有稳定的股利政策,并且股利支付次数多、股利支付率也较高。当上市公司缺
乏良好的投资机会并且具有足够的现金流时,公司治理系统可以迫使经理人向股东分配
红利,股东分配红利则间接杜绝了内部人为一己私利攫取公司利润,从而进一步保护了
股东的利益。

美国上市公司 1983 年至 1992 年间的平均股利支付率达到65.8\%
,\cite{李常青1999我国上市公司股利政策现状及其成因}也就是说这些上市公司大部分
净利润都通过股利的形式回报了股东,至今,虽然股利支付率有一定程度的下降,但他
们有维持着股利政策的连续性,并且均以现金股利为主要的支付方式。

相比之下,我国的公司治理系统则缺乏足够的效率,难以约束上市公司主动发放股利回
报投资者。早在 A 股市场建立之初,上市公司的股利支付率普遍偏低甚至很多公司都不
分红。20 世纪 90 年代后期,进行派现的上市公司比例和派现水平都呈下降趋势——现金
分红公司仅占 30\%左右,\cite{李常青2001股利政策理论与实证研究} 现金股利占净利
润的比例低于 30\% 而现金股利的市场回报不到 1\%。为了提升市场对投资者的回报,
交易所积极鼓励和推动上市公司完善现金分红政策,继第 57 号令之后,2013 年初发布
《上海证券交易所上市公司现金分红指引》,表明了现金分红监管政策的制定和执行是
当前上海证券交易所重点关注的业务领域之一。又如深圳证券交易所的交易规则规定:
上市公司应当重视对投资者特别是中小投资者的合理投资回报,制定持续、稳定的利润
分配政策。由于近年来监管层政策的日臻完善和舆论监督的压力,上市公司对利润分配
愈加重视。在这种背景下,实证检验市场和投资者对上市公司现金红利和股票股利预案
的态度,为股利政策的出台和推动上市公司分配股利提供了重要支撑,对于保护股东的
股利分配权、培育投资者的长期投资理念以及增强资本市场的吸引力和活力具有重要意
义。\cite{Avner1982Stockholder,Kryzanowski2009Trading,Walther2007Stock,
  CHRISTOPHER1987The,Mcnichols1990Stock,Merton1961Dividend,Hong2012Trading,
  Graeme1996Stice}

对股利政策的研究中,涉及到股票股利的理论主要有如下:

\begin{description}
  \item[股利信号传递理论]公司分配股票股利与公司对未来收益的预期有关。即,公司
    为了让投资者确认公司的未来收益将会增加,利用的是公司与投资者的信息不对称
    。\cite{Josef1986Tax,Gerald1992Jensen,Eades1984On,
    Measurement1998the,Vermaelen1983Tax,Lintner1956DISTRIBUTION}
  \item[交易区间理论]公司分配股票股利与股票的可流动性与可销售性有关。即,公司
    基于过去的股票信息决定是否分配股票股利,以此来保证股价能维持在一定的交易
    区间。\cite{Lintner1964Optimal,R2012DIVIDEND,Hersh1984Explaining,
    Stein1989Efficient,Pyung1995Signaling}
  \item[自我选择假说]公司分配股票股利与公司对未来收益的预期和有关过去的股票信
    息都有关。\cite{Sheikh1989Stock,J2014STOCK,Woolridge1983Randall,
    Copeland1979Liquidity}
  \item[税收选择理论]公司出于税后收益最大化分配股票股利。通常当股息收入的个人
    所得税低于资本利得的个人所得税时,公司将向股东支付股息
    。\cite{Mozes1995THE,Muscarella1996Stock,Rozeff1982GROWTH,Bar1977A,
    Paul2000Stock}
\end{description}

不同的实证研究\cite{姜国华2006公司治理和投资者保护研究综述,
  黄娟娟2007上市公司的股利政策究竟迎合了谁的需要——来自中国上市公司的经验数据,
  何涛2002现金股利能否提高企业的市场价值——,
  杨鹏2012创业板公司股利分配对未来业绩影响探析,
  方光正2005上市公司偏好低现金股利政策的成因分析,
  陈晓1998我国上市公司首次股利信号传递效应的实证研究,
  陈珠明2010高送转股票财富效应的实证研究,
  曹媛媛2004我国上市公司股利政策的信息内涵,
  曹玉珊2008企业财务可持续增长效率的源泉分析——来自中国上市公司的证据,
  周县华2008股权分置改革、高股利分配与投资者利益保护——基于驰宏锌锗的案例研究,
  张海燕2008从现金红利看第一大股东对高级管理层的监督,
  应展宇2004股权分裂、激励问题与股利政策——中国股利之谜及其成因分析,
  杨熠2004现金股利,
  谢军2006Dividend,
  肖珉2005Free,
  俞乔2001我国公司红利政策与股市波动,
  尹飘扬2012我国创业板上市公司股利分配研究,
  易颜新2008我国上市公司股利分配决策的调查研究分析,
  吴斯2011中小板上市公司股利决策羊群行为研究,
  吴苗苗2011创业板上市公司的“高送转”股利政策研究,
  翁洪波2007机构投资者、公司治理与上市公司股利政策,
  魏刚2001A,
  魏刚2001中国上市公司股利分配问题研究,
  王会芳2011创业板上市公司股利分配研究,
  汪平2009资本成本、可持续增长与国有企业分红比例估算——模型构建及检验,
  沈洪涛2003新股增发,
  秦江萍2004中国上市公司现金分红实证分析,
  任有泉2006An,
  平新乔2003中国国有企业代理成本的实证分析,
  彭洋创业板上市公司股利分配政策传递的信号分析,
  吕长江2002上市公司资本结构、股利分配及管理股权比例相互作用机制研究,
  陈沛歆2012股权分置改革、资金侵占与现金股利政策——基于,
  吕长江2005公司治理结构与股利分配动机,
  罗宏2008国企分红、在职消费与公司业绩,
  罗宏2008国企分红、在职消费与公司业绩,
  陆正华2010创业板公司,
  刘淑莲2003中国上市公司现金分红实证分析,
  廖理2004管理层持股,
  廖理2005股利政策代理理论的实证检验,
  李光贵2009国有控股上市公司现金分红行为,
  李春玲2009控股股东与上市公司股利政策
}对不同的理论均有不同程度的支撑。

考虑到如果公司是根据过去的股票信息决定分配股票股利,那么我们就可以根据已得的
信息预测未来公司的股票是否会高送转。预测的准确性取决于理论是否适用于当前的
情况。

\section{问题重述}%
\label{sec:problem}

\begin{Exercise}[label = ex:1]
  根据2011年到2018年的数据,利用经济学和统计学方法,建立模型来预测哪些上市公
  司会实施高送转,分析导致对上市公司实施高送转有较大影响的可能原因。
\end{Exercise}

\section{研究内容}%
\label{sec:content}

\begin{figure}[htbp]
  \centering
  \includegraphics[
    width = 0.8\linewidth,
  ]{fig/content}
  \caption{研究内容}%
  \label{fig:content}
\end{figure}

\end{document}

