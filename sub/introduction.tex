\documentclass[../main]{subfiles}
\begin{document}

\chapter{绪论}%
\label{cha:introduction}

\section{问题背景}%
\label{sec:backend}

主板以及中小企业板的高送转股票在近几年日益盛行,导致了创业板公司更加注重高送
转。但信息不对称现象使得高送转股票容易涉及内幕交易。包括:

\begin{itemize}
  \item 拉抬上市公司股价,为上市公司的增发做铺垫;
  \item 帮助套牢机构实施突围;
  \item 掩护解禁股套现。
\end{itemize}

经证监会披露,高送转占涉及内幕交易的案件的比例高达3成以上。即便在证监会对涉嫌
市场操纵、内幕交易等违法行为的高送转公司加大了立案调查的力度的严厉打击下,凭
借自身掌握或获取的信息优势谋取非法利益, 违背了诚信履行受托责任,义务,严重破
坏资本市场信息公平原则和影响证券市场价格发现和资源配置功能的内幕交易人员仍旧
层出不穷。这是证监会加强市场监管工作的一个重大考验——证监会能否继续维持对内幕
交易等违法违规行为的零容忍。

在这种背景下,分析公司分配高送转股票股利的动机,研究市场和投资者对于更多的分
配股利的公司、更高比例的送转股水平的态度,有利于监管部门合力预期和把握后续对
市场新政的制定和出台有着至关重要的意义
。\cite{卢玉梅2014我国上市公司“高送转”行为动机实证研究}

\section{问题重述}%
\label{sec:problem}

\begin{Exercise}[label = ex:1]
  根据2011年到2018年的数据,利用经济学和统计学方法,建立模型来预测哪些上市公
  司会实施高送转,分析导致对上市公司实施高送转有较大影响的可能原因。
\end{Exercise}

\end{document}

