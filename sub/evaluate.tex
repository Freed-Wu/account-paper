\documentclass[../main]{subfiles}
\begin{document}

\chapter{评价}%
\label{cha:evaluate}

\section{缺点}%
\label{sec:weakness}

\begin{description}
  \item[评估模型方法需要优化]使用Logistic和SVM模型进行预测时对预测结果的评估
    使用的只是排序打分方法,但其评估效果并不够完善,在后续中可以考虑使用混淆
    矩阵进行评估。混淆矩阵能够更加直观的给出模型的准确率、精确度、灵敏度和特
    异度,能够对两个预测模型的融合的效果更好;
  \item[因子池需要拓宽]影响上市公司实施高转送的因子有很多,但由于条件所限,只
    能从中选取部分因子进行分析,在后续中可以进一步增加因子的个数。
\end{description}

\section{展望}%
\label{sec:vista}

\begin{itemize}
  \item 选用更好的评价模型;
  \item 采用更多的因子。
\end{itemize}

\end{document}

