\documentclass[../main]{subfiles}
\begin{document}

\chapter{理论分析}%
\label{cha:theorem}

\section{相关定义}%
\label{sec:definition}

\subsection{股利}%
\label{sec:dividend}

公司向股东分配的公司盈余。在全球各个国家的相关法律规定几乎相同
。\cite{孔小文2003上市公司股利政策信号传递效应的实证分析}

\begin{itemize}
  \item 在欧美国家,股利只能从公司历年盈余中支付。公司只有在具备足够的累积盈
    余时才能支付股利;
  \item 在我国,法律允许将超额面值缴入资本,也即资本公积(Additional Paid-in
    Capital)列入可供股东分配的范围。
\end{itemize}

公司股东大会或董事会制定对一切与股利有关的事项,即股利政策,包括:

\begin{itemize}
  \item 公司是否发放股利,即对其收益进行分配还是留存以用于再投资的策略问题;
  \item 股利的支付方式;
  \item 发放多少股利;
  \item 何时发放股利。
\end{itemize}

在我国,上市公司股利支付方式有多种:

\subsection{现金股利}%
\label{sub:cash_dividend}

以现金方式向股东发放的红利,最常见。必须具备条件:

\begin{itemize}
  \item 有足够的留存收益,以保证再投资资金的需要;
  \item 有足够的现金,以保证生产经营和支付股利;
  \item 有利于改善公司的财务状况
  \item 有董事会的决定。
\end{itemize}

现金红利的支付会直接影响股票的市场价格,公司必须依据实际情况权衡利弊,制
定合理的现金红利政策。

\subsection{股票股利}%
\label{sub:stock_dividend}

以增发股票作为股利。按原有普通股东的持股比例分配新股,其本质是等比例增加原有
股东持有流通股的数量,稀释每股普通股的权益,而不会影响公司的资产负债,也不会
增加股东权益的总额。股票股利实质是股东权益的内部结构调整,对净资产收益率没有
影响,对公司的盈利能力也并没有任何实质性影响。根据支付股票股利的资产来源,股
票股利又分为:

\begin{description}
  \item[送红股]将净利润以发放股票的形式分配给股东,结果是利润转为股本。送红股
    后,公司的资产、负债、股东权益的总额及结构并没有发生改变,但总股本增大了
    ,同时每股净资产降低了;
  \item[转增股票]将盈余公积金和资本公积金转换为股本,并以发放股票的形式分配给
    所有股东。盈余公积金是指企业按照规定从税后利润中提取的积累资金,主要用来
    弥补企业以往年度的亏损和转增资本。资本公积金是在公司的生产经营之外,由资
    本、资产本身及其他原因形成的股东权益收入。公司的资本公积金主要来源于的股
    票发行的溢价收入、接受的赠与、资产增值、因合并而接受其他公司资产净额等。
\end{description}

高送转指上市公司大比例送红股或大比例以资本公积金转增股票,一般每10股送红股和
转增股票大于等于5 。

\section{文献综述}%
\label{sec:document}

\subsection{理论研究}%
\label{sub:theorem}

投资者是资本市场重要的参与者,保护投资者利益、积极回报投资者是资本市场的立身
之本。在成熟的资本市场中,完善的投资者法律保护机制、有效的市场约束机制以及较
为健全的公司治理机制,为保护股东利益提供了强有力的支撑。这些经济体中的上市公
司大多具有稳定的股利政策,并且股利支付次数多、股利支付率也较高。当上市公司缺
乏良好的投资机会并且具有足够的现金流时,公司治理系统可以迫使经理人向股东分配
红利,股东分配红利则间接杜绝了内部人为一己私利攫取公司利润,从而进一步保护了
股东的利益。

美国上市公司 1983 年至 1992 年间的平均股利支付率达到65.8\%
,\cite{李常青1999我国上市公司股利政策现状及其成因}也就是说这些上市公司大部分
净利润都通过股利的形式回报了股东,至今,虽然股利支付率有一定程度的下降,但他
们有维持着股利政策的连续性,并且均以现金股利为主要的支付方式。

相比之下,我国的公司治理系统则缺乏足够的效率,难以约束上市公司主动发放股利回
报投资者。早在 A 股市场建立之初,上市公司的股利支付率普遍偏低甚至很多公司都不
分红。20 世纪 90 年代后期,进行派现的上市公司比例和派现水平都呈下降趋势——现金
分红公司仅占 30\%左右,\cite{李常青2001股利政策理论与实证研究} 现金股利占净利
润的比例低于 30\% 而现金股利的市场回报不到 1\%。为了提升市场对投资者的回报,
交易所积极鼓励和推动上市公司完善现金分红政策,继第 57 号令之后,2013 年初发布
《上海证券交易所上市公司现金分红指引》,表明了现金分红监管政策的制定和执行是
当前上海证券交易所重点关注的业务领域之一。又如深圳证券交易所的交易规则规定:
上市公司应当重视对投资者特别是中小投资者的合理投资回报,制定持续、稳定的利润
分配政策。由于近年来监管层政策的日臻完善和舆论监督的压力,上市公司对利润分配
愈加重视。在这种背景下,实证检验市场和投资者对上市公司现金红利和股票股利预案
的态度,为股利政策的出台和推动上市公司分配股利提供了重要支撑,对于保护股东的
股利分配权、培育投资者的长期投资理念以及增强资本市场的吸引力和活力具有重要意
义。\cite{Avner1982Stockholder,Kryzanowski2009Trading,Walther2007Stock,
  CHRISTOPHER1987The,Mcnichols1990Stock,Merton1961Dividend,Hong2012Trading,
  Graeme1996Stice}

对股利政策的研究中,涉及到股票股利的理论主要有如下:

\begin{description}
  \item[股利信号传递理论]公司分配股票股利与公司对未来收益的预期有关。即,公司
    为了让投资者确认公司的未来收益将会增加,利用的是公司与投资者的信息不对称
    。\cite{Josef1986Tax,Gerald1992Jensen,Eades1984On,
    Measurement1998the,Vermaelen1983Tax,Lintner1956DISTRIBUTION}
  \item[交易区间理论]公司分配股票股利与股票的可流动性与可销售性有关。即,公司
    基于过去的股票信息决定是否分配股票股利,以此来保证股价能维持在一定的交易
    区间。\cite{Lintner1964Optimal,R2012DIVIDEND,Hersh1984Explaining,
    Stein1989Efficient,Pyung1995Signaling}
  \item[自我选择假说]公司分配股票股利与公司对未来收益的预期和有关过去的股票信
    息都有关。\cite{Sheikh1989Stock,J2014STOCK,Woolridge1983Randall,
    Copeland1979Liquidity}
  \item[税收选择理论]公司出于税后收益最大化分配股票股利。通常当股息收入的个人
    所得税低于资本利得的个人所得税时,公司将向股东支付股息
    。\cite{Mozes1995THE,Muscarella1996Stock,Rozeff1982GROWTH,Bar1977A,
    Paul2000Stock}
\end{description}

\subsection{实证研究}%
\label{sub:proof}

不同的实证研究\cite{姜国华2006公司治理和投资者保护研究综述,
  黄娟娟2007上市公司的股利政策究竟迎合了谁的需要——来自中国上市公司的经验数据,
  何涛2002现金股利能否提高企业的市场价值——,
  杨鹏2012创业板公司股利分配对未来业绩影响探析,
  方光正2005上市公司偏好低现金股利政策的成因分析,
  陈晓1998我国上市公司首次股利信号传递效应的实证研究,
  陈珠明2010高送转股票财富效应的实证研究,
  曹媛媛2004我国上市公司股利政策的信息内涵,
  曹玉珊2008企业财务可持续增长效率的源泉分析——来自中国上市公司的证据,
  周县华2008股权分置改革、高股利分配与投资者利益保护——基于驰宏锌锗的案例研究,
  张海燕2008从现金红利看第一大股东对高级管理层的监督,
  应展宇2004股权分裂、激励问题与股利政策——中国股利之谜及其成因分析,
  杨熠2004现金股利,
  谢军2006Dividend,
  肖珉2005Free,
  俞乔2001我国公司红利政策与股市波动,
  尹飘扬2012我国创业板上市公司股利分配研究,
  易颜新2008我国上市公司股利分配决策的调查研究分析,
  吴斯2011中小板上市公司股利决策羊群行为研究,
  吴苗苗2011创业板上市公司的“高送转”股利政策研究,
  翁洪波2007机构投资者、公司治理与上市公司股利政策,
  魏刚2001A,
  魏刚2001中国上市公司股利分配问题研究,
  王会芳2011创业板上市公司股利分配研究,
  汪平2009资本成本、可持续增长与国有企业分红比例估算——模型构建及检验,
  沈洪涛2003新股增发,
  秦江萍2004中国上市公司现金分红实证分析,
  任有泉2006An,
  平新乔2003中国国有企业代理成本的实证分析,
  彭洋创业板上市公司股利分配政策传递的信号分析,
  吕长江2002上市公司资本结构、股利分配及管理股权比例相互作用机制研究,
  陈沛歆2012股权分置改革、资金侵占与现金股利政策——基于,
  吕长江2005公司治理结构与股利分配动机,
  罗宏2008国企分红、在职消费与公司业绩,
  罗宏2008国企分红、在职消费与公司业绩,
  陆正华2010创业板公司,
  刘淑莲2003中国上市公司现金分红实证分析,
  廖理2004管理层持股,
  廖理2005股利政策代理理论的实证检验,
  李光贵2009国有控股上市公司现金分红行为,
  李春玲2009控股股东与上市公司股利政策
}对不同的理论均有不同程度的支撑。

\subsubsection{国外}%
\label{ssub:foreign}

Lintner(1956)对来自 28 家公司的经理进行访谈后发现,盈余水平是股利变动的最重
要的决定因素,股利政策是“粘性”的,与公司长期可持续盈余紧密相关,它往往被成熟
公司采用,而且存在着各年之间的“平滑”现象。该种观点随后受到了众多学者的检验,
并在不同的国家和地区得到了证实。

Higgins(1972)认为影响股利的最主要因素是预期增长率、投资机会、财务杠杆以及各
种风险,股利支付率上升会使得公司投资机会所需资金减少,反之亦然。Smith 和
Watts(1992)则认为行业因素也会影响股利政策的制定。Jensen 和 Meckling(1976)
的检验表明,增长机会与股利支付率呈负相关关系,即公司可以支付股东较少的现金红
利,而把更多的现金留给增长投资。Rozeff(1982)对此观点也表示赞同,他表示公司
减低股利分配是它预计到收入快要达到较高的增长速度,而较高的增长机会往往需要较
高的资金投入,因此无法承担原先的股利支付水平。Masulis 和 Trueman(1988)认为
高增长的公司都希望能投入所有的内部资金在使公司价值增长的项目上,并不像成熟公
司迫于内部资金无法在投资项目上用尽而只能发放股利。 Baker 等于 1985 年对上市公
司管理层调查后表示,不同于 Higgins(1972)的研究,对股利政策起决定性作用的因
素是:未来的盈利能力、历史股利政策、现金流量的水平以及提高还是稳定股价的策略
。

Mancinelli 和 Ozkan(2006)通过研究发现,公司股权结构的特征会影响股利政策的制
定。美国纳斯达克市场在制定股利政策时考虑了一些影响极大而且重要的决策因素,
Kent 等(2001)总结这些股利决策的重要因素为:过去的股利政策、收益的稳定性、最
近的收益水平和预期收益水平。这些在纳斯达克里有重要地位的因素,对于美主板也同
样重要,也再次证明纳斯达克上市公司的管理者一直沿用 Lintner(1956)的现金股利
理论来考虑股利政策。

Fama 和 French(2001)的结论不可忽视。他们认为影响股利政策的因素有:盈余水平
、投资机会和公司规模。对此,他们通过研究 1926 年至 1999 年期间的样本公司股利
分配水平,重点分析了 1972 年至 1999 年期间的情况后发现,分配股利公司的比例于
1978 年后出现下降,从 1978 年的 65%左右大幅下降至 1999 年的 21%。而这种情况的
发生是由于分配股利的样本公司的公司特征出现了变化,具体地讲就是这些样本公司伴
有规模较小、收益较低且拥有富余的投资机会等特征。同理,研究也从未派现样本公司
中找到了类似的典型特征。根据上述结果,Fama 和 French(2001)得出观点:企业会
考虑自身的企业特征再进行股利支付。于是,公司不管有无良好的投资机会,其现金股
利公告传递了公司的投资策略,在投资机会较好的情况下,公司就不愿意分配现金给股
东,相反,如果公司公告其股利分配方案可能是因为投资机会的缺乏(Baker(1989),
Brook 等(1998),Baker 和 Wurgler(2003),

Pan(2001)),Brav 等(2003)对上市公司首席财务官如何看待股利支付受投资机会
影响的问题上进行了信息采集,结果显示接近 50%的财务管理人员赞同投资决策是否足
够好而且是否容易取得是非常重要的股利政策影响因素。除了上面所述的因素,另一个
影响公司股利分配的是公司规模的大小。大公司较为容易获得资本市场的融资,减轻了
不少对内部留存的依赖,故而可以实施高水平的现金支付。这一观点也得到了 Holder
等(1998),Lloyd 等(1985),Ambarish 等(1987),Vogt(1994)等的支持。比如
,Ahmad 和 Carlos(2008)研究美国制造业后表示,派现公司与未派现公司在整体上进
行比较,通常有更高的资金流动性、更好的盈利水平、更大的公司规模以及更大力度的
研发投入等公司特征,该结论与 Fama 和 French(2001)的观点也是一致的。Ambarish
等(1987)则是从创业板公司规模的角度来看,他们认为公司价值较高的创业板上市公
司在能够快速发展,但处于公司发展的中间阶段时,就如他们积极地进行投资一样也很
乐意来支付现金股利,以此来将自己与价值较低的公司区分开来。

根据相关研究,公司的财务风险也会影响到股利政策的制定。Rozeff(1982)发现,财
务风险突出的公司由于财务杠杆高,为了降低外部融资成本的压力而更可能支付较少的
股利。DeAngelo 和 DeAngelo(1990)对有财务风险的样本公司进行研究,表明有 1/3
的样本公司并未降低股利水平。为了替代有成本压力的外部融资来对股利进行支付,这
些样本公司将寻找其他的融资方式。Rajiv Sant 和 Arnold R. Cowan(1999)的研究表
明公司一旦进行现金分红,表示该公司预期未来收益与现金流足以用于股利支付,但对
于成长性公司而言,由于对未来收益与现金流的不确定,所以并不愿意支付股利。
Graham(1985)在考察了股利、投资与融资的关系后认为,在融资处于受限制状态时,
股利和投资政策不是独立的,而必须将融资决策因素考虑进去。 Jensen 和 Meekling(
1976)的研究认为,公司的增长机会较多时,股东们可以接受较低的现金股利率,增长
机会和股利支付水平成负相关。

\subsubsection{国内}%
\label{ssub:national}

在研究“高送转”的市场效果时,通常“事件研究法”。即选定上市公司“高送转”预案公告
日作为事件日(即 t=0),在事件窗口内观察“高送转”发生的前后一段时间股票价或报
酬率的变化,从而得出其带来的市场影响。关于股利政策的信号传递理论,魏刚(1998
)认为我国上市公司的股利政策向投资者传递了公司持久盈利的信息,符合信号传递理
论;何涛和陈小悦(2003)研究了我国上市公司的送转股动机,发现信号传递假说和流
动性假说对我国的送转股行为解释能力有限,送转股公司送转之后的盈利并未实现更好
的增长,并提出了价格幻觉假说来解释我国上市公司的送转股行为。冯科、刘宏和何理
(2012)通过建立线性模型来得出高送转对盈利的信号效应,通过高送转对上市公司未
来盈利的信号效应的实证分析,可以得出结论:(1)每股股价越高的上市公司越有可能
进行高送转;每股股价涨幅越大的上市公司越有可能进行高送转;每股资本公积和盈余
公积越高的上市公司越有可能进行高送转;M上市公司的高送转行为与市值大小没有显
著关系。(2)现金红利增加的高送转公司 Spfac 系数为正且在 10%的置信度下显著,
这说明支付更高现金红利的同时进行高送转是一个可信的上市公司未来具有良好盈利的
信号。但是,在其他情况下上市公司进行高送转并没有传递上市公司管理层关于未来经
营业绩会变得更好的私人信息,这与股票股利的信号效应假说不一致。

彭洋(2011)则基于创业板“高送转”的情况研究发现,虽然有的创业板公司净利润账面
数据没有出现明显下滑,净利润增长水平较往年却有大幅度的下降,则说明该创业板公
司增长动力不足,成长能力存在问题。毫无疑问,净利润率大幅下降的情况使得公司减
少股利分配,传递的是不好的消息,结果可能导致公司股价下跌。相反的是,如果创业
板公司加大了研发投入力度,那么股利的下降传递的就不是负面信号,而是公司未来发
展能力变强的信号,再加上上市初期过高的市盈率转回、公司价值合理回归。这样对股
价会有正面积极的影响作用。总而言之,创业板公司的股利政策对于投资者的影响具有
不确定性,股利支付率的提高不一定是好的事情,而股利支付率的下降也不一定是坏的
信号。股利水平变化传递的信号是好还是坏则需要具体分析股利分配方案背后所传递的
公司财务情况以及公司未来的投资经营前景。也就是说,只有看清公司是真的高成长、
高增长业绩的还是低成长、业绩较差的,才能够决定该股利信息是好还是坏。

此外,徐慧玲,吕硕夫(2012)运用多元线性回归的方法,基于我国股权的结构特征进
行分析,得出如下结论:上市公司流通股比例与“高送转”股利政策负相关;上市公司前
十大股东持股比例与“高送转”股利政策正相关;市公司机构持股比例与“高送转”股利政
策负相关。

当然,也有研究发现高送转政策可能会传达错误的信号。吴苗苗(2011)指出,当大多
数中小股民受 2011 年创业板股价全体大跌而平均亏损额度达 20% 以上时,才发现实施
高送转股利分配的创业板公司的真正目的,然而为时已晚。正是如此,在创业板市场股
息率不高、非现金分红仍较为普遍的背景下,研究这类股利分配方式具有一定的现实意
义(吴苗苗,2011)。

此外,李慧宇(2012)根据创业板市场的高风险与信号传递的关系发表了看法,他表示
创业板市场不太可能传递出能够让投资者获得稳定收益的信号,结果导致投资者更加的
投机,“打了就跑”的策略普遍存在。现实情况表明,通常状况下创业板公司股价存在较
大幅度波动,强制分红政策的实施,未必不能稳定投资者的预期回报。基于此,创业板
股票价格或许能够回归合理,并维持在正常的区间和浮动程度内。

陈晓、陈小悦和倪凡(1998)在信号传递理论背景下研究了 1996 年以前上市的公司的
首次红利分配政策,并采用累计超额收益率法得出股价在首次股利公告日前后各 20 天
会产生平均超额收益率的结论。俞乔、程滢(2001)以 1992 年到 2000 年期间沪深两
市所有发放股利的上市公司为研究样本,研究结果指出,分红公告日当天及公告日后一
天的股价都出现了显著的超额收益,市场对送股或混合股利的分红政策有较强的正向反
应,市场对单纯现金红利则不敏感。陆正华(2010)得出的结论认为:创业板中的高送
转公司可以获得超额收益。陈珠明、史余森(2010)的实证结果表明,在公司公告分红
预案前 5 个交易日内,高送转股票会出现显著正的超额收益;牛市中除权除息日前后的
股票会在短期内存在超额收益;熊市中除权除息日前后的超额收益不显著。此外,邓雄
博和刘锡标(2013)发现本文研究表明,我国的证券市场,存在高送转预案公布日效应
,尤其在高送转预案公布日前 7 个交易日存在显著的正超额收益率,而预案公布日之后
,正超额收益率不明显并且投资风险加大;田丰(2014)通过对 A 股高送转市场反应的
分析,发现预案公告日前的高送转效应不显著,而预案公告日后的高送转效应显著。

\section{研究内容}%
\label{sec:content}

考虑到如果公司是根据过去的股票信息决定分配股票股利,那么我们就可以根据已得的
信息预测未来公司的股票是否会高送转。预测的准确性取决于理论是否适用于当前的
情况。

\begin{figure}[htbp]
  \centering
  \includegraphics[
    width = 0.6\linewidth,
  ]{fig/content}
  \caption{研究内容}%
  \label{fig:content}
\end{figure}

\end{document}

